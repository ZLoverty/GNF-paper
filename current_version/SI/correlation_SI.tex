\documentclass[twocolumn,aps,prx,amsmath,amssymb,longbibliography]{revtex4-2}
\usepackage{graphicx}
\usepackage{dcolumn}
\usepackage{bm}
\usepackage{amsfonts}
\usepackage{xcolor,tabu}
\usepackage{multirow}
\usepackage{amsthm}
\usepackage{textcomp}
\usepackage{tikz}
\usepackage[colorlinks=true,
            linkcolor=blue,
            urlcolor=blue,
            citecolor=blue]{hyperref}
\hypersetup{bookmarksopen=true}
\usepackage{xr}


% \author{Zhengyang Liu, Wei Zeng, Xiaolei Ma, Xiang Cheng}
%\email{liux3141@umn.edu}
% \affiliation{Department of Chemical Engineering and Materials Science, University of Minnesota, Minneapolis, Minnesota 55455, USA}
\date{\today}

\begin{document}
\title{Density Fluctuations and Energy Spectra of 3D Bacterial Suspensions\\
        Supplemental Material}


\author{Zhengyang Liu}
%\email{liux3141@umn.edu}
\author{Wei Zeng}
\author{Xiaolei Ma}
\author{Xiang Cheng}


\affiliation{Department of Chemical Engineering and Materials Science, University of Minnesota, Minneapolis, MN 55455, USA}


\maketitle

% What is needed in this supplemental material?
% - all the methods I have used in the main text, including:
%   - bacterial samples
%   - microscopy
%   - GNF
%   - PIV
%   - cross-correlation
%   - energy spectra
% - Justification of methods and parameter choices
%   - two GNF methods
%   - local density flucution duration (10 frames)
%   - kinks in energy spectra curves
%   - normalize GNF at small length scale
%   - does image contrast matter?
% - Movie
%   - vigorous bacterial turbulence
%   - PIV overlay

\section{Materials and methods}
\subsection{Light-powered \textit{E. coli}}
We introduce a light-driven transmembrane proton pump, proteorhodopsin (PR), to wild-type \textit{E. coli} (BW25113) by transforming the bacteria with plasmid pZE-PR encoding the SAR86 $\gamma$-proteobacterial PR-variant (Walter 2007). The activity of PR is correlated with the intensity of light. Thus, we can control the swimming speed of bacteria using light of different intensities. In our experiments, we use high-intensity light, which saturates the light response of bacteria. The average swimming speed of bacteria is fixed at $v_0 = 15 \pm 3$ $\mu$m/s in the dilute limit.

The bacteria are cultured at 37 \textcelsius{} with a shaking speed at 250 rpm for 14-16 hours in terrific broth (TB) [tryptone 1.2\% (w/w), yeast extract 2.4\% (w/w), and glycerol 0.4\% (w/w)] supplemented with 0.1 g/L ampicillin. The culture is then diluted 1:100 (v:v) in fresh TB and grown at 30 \textcelsius{} for 6.5 hours. PR expression is triggered by supplementing the culture medium with 1 mM isopropyl $\beta$-D-thiogalactoside and 10  $\mu$M ethanolic all-trans-retinal in the mid-log phase, 3 hours after the dilution.

The bacteria are harvested by gentle centrifugation ($800g$ for 5 min). After discarding the culture medium in the supernatant, we resuspend bacteria with DI water. The resuspended suspension is then centrifuged again at $800g$ for 5 min, and finally adjusted to the target concentration for experiments.

\subsection{Sample preparation and microscopy}

To prepare the sample for microscopy, we construct a seal chamber made of glass slides (25 mm $\times$ 75 mm) and coverslips (18 mm $\times$ 18 mm). We first glue (NOA 81, Norland, NJ) two coverslips on a glass slide, side-by-side, leaving a 3-mm separation between the two coverslips. We then cover the 3-mm separation with another coverslip to form a channel. We then pipette bacterial suspensions into the channel. Finally, we seal the two ends of the channel using UV glue (NOA 76, Norland, NJ) to form a sealed chamber.

Images of the bacterial suspensions are taken 50 $\mu$m above the bottom surface of the sealed chamber by a Nikon Ti-E inverted microscope using the bright field mode with a 20$\times$ (NA 0.5) objective. The field of view is 420 $\times$ 360 $\mu$m$^2$. All videos are recorded at 30 frames per second using a sCMOS camera.

%%\subsubsection{Linear relation between intensity and density}
%%In the low attenuation limit (small thickness and weak absorptivity), $I=I_0-\epsilon$ where $\epsilon\ll I_0$:
%%\begin{equation}
%%  \log \frac{I_0}{I} = \log \frac{I_0}{I_0 - \epsilon} \approx \frac{\epsilon}{I_0} \sim c
%%\end{equation}
%%\begin{equation}
%%I = I_0 - \epsilon \sim c
%%\end{equation}
%%where $I_0$ is the original light intensity, $I$ is the transmission light intensity, $\epsilon$ is the attenuated light intensity, which is much smaller than the original light intensity.

%%%%%%%%%%%%%%%%%%%%%%%%%%%%%%%%%%%%%%%%%%%%%%%%%%%%%%%%%%%%%%%%%%%%%%%%%%%%%%%%%%%%%%%%%%%

\begin{figure*}[!]
\begin{center}
\includegraphics[width=0.8\textwidth]{figures/GNF-calculation/v1.pdf}
\caption[GNF calculations]
{
\textbf{GNF calculations.}
(a) Varying subsystem sizes.
(b) Multiple seeds of subsystems for spatial average.
}
\label{GNF-calculation}
\end{center}
\end{figure*}


\section{Image analysis}
\subsection{Density fluctuations}
\label{sec:GNF-calculations}
% \subsubsection{Two different methods}
%
% Two different methods have been used to calculate number density fluctuations in experiments \cite{Narayan2007,Aranson2008}. Both methods divide a large system into small subsystems of fixed size $l$. The first method calculates the mean particle number $N$ and the standard deviation of particle number $\Delta N$ in each subsystem over time first and then average $N$ and $\Delta N$ spatially over all the subsystems, i.e., temporal variance followed by spatial average (the TVSA method) \cite{Narayan2007}. The second method reverses the procedure, which calculates $N$ and $\Delta N$ over all the small subsystems in one frame first and then takes an temporal average over all frames, i.e., spatial variance followed by temporal average (the SVTA method) \cite{Aranson2008}.
%
% \textcolor{red}{I don't understand the purpose of this paragraph. Do you like to argue that the two methods are the same for our system? But this is not correct.} Ref. XX showed that when a system is homogeneous, where spatial and temporal correlations are small compared with the system size and experiment duration, the two methods lead to the same results. Bacterial suspensions used in our study have correlation lengths smaller than the system size ($\sim 140$ \textmu m), and is thus spatially homogeneous. When the duration of experimental videos is longer than the autocorrelation times, the system can be also regarded as being temporally homogeneous as well.
%
% In our experiments, we correlate image pixel intensity with local particle density. Nevertheless, the non-uniformity of light illumination induces temporally stable intensity variations even without bacterial suspensions. Such intrinsic intensity variations would lead systematic errors in density fluctuations when the second method of SVTA is used. Thus, we adopt the TVSA method, which is insensitive to the temporally stable intensity variations and avoids the potential errors due to the non-uniform light illumination.




%%%%%%
\subsubsection{Pixel intensity and bacterial number}
In Fig.~1d, we show that under the same illumination and imaging condition, bacterial density and the average pixel intensity follow approximately a linear relation, which can be expressed as follows:
\begin{equation}
  \label{eq:phi-I-relation}
  \phi = a + bI,
\end{equation}
where $\phi$ is the volume fraction of bacterial suspensions, $I$ is the average pixel intensity, $a$ and $b$ are constants under the same illumination and imaging condition. The number of bacteria in a given subsystem with side length $l$ and thickness $d$ can be calculated as
\begin{equation}
  \label{eq:n}
  N = \frac{l^2d}{V_b} \phi = \frac{l^2d}{V_b} (a+bI),
\end{equation}
where $V_b$ is the volume of a single bacterium. \textcolor{red}{$d$ is the depth of the field of microscopy, which is fixed in our experiments. We estimate $d \approx XX$ $\mu$m. Yi approximated it to be about 10 $\mu$m. But I am not sure if he has any solid evidence on the number.} Thus, the number of bacteria in the subsystem $N$ is proportional to $l^2 \phi$. Taking the standard deviation of both sides of Eq.~\ref{eq:n}, we get
\begin{equation}
  \label{intensity-number}
  \Delta N = \frac{l^2 d}{V_b}|b|\Delta I,
\end{equation}
where $\Delta N$ and $\Delta I$ are the standard deviation of the bacterial number and the standard deviation of the average pixel intensity of the subsystem over time, respectively. Since $d|b|/V_b$ is a constant independent of subsystem sizes and bacterial volume fractions, $\Delta N$ is linearly proportional to $l^2\Delta I$. Because any constant in front of $\Delta N$ would not affect either the scaling relation or the relative magnitude of density fluctuations at different $\phi$, we simply use $l^2\Delta I$ as $\Delta N$ in our study.

%%%%%%%%

\begin{figure}[t]
	\begin{center}
		\includegraphics[width=0.46\textwidth]{Figures/GNF-normalization/same-illumination.pdf}
		\caption[Density autocorrelation]
		{
			\textbf{GNF curves at various volume fractions under same illumination and imaging conditions.}
		}
		\label{fig:same-conditions}
	\end{center}
\end{figure}

\subsubsection{Density fluctuations at different length scales}

Based on the linear relation between $\Delta N$ and $\Delta I$, we calculate the density fluctuations at different length scales. We first crop square-shape subsystems of increasing sizes, as shown in Fig.~\ref{GNF-calculation}a. For each subsystem size $l$, the standard deviation of the average pixel intensity of the subsystem is calculated over 50 frames (1.67 s or 8.35$\tau_b$), which is longer than the saturated density correlation time (Fig. 2f). To improve statistics, we choose 20 subsystems of the same size evenly distributed in the field of view and obtain a spatial average of the temporal standard deviation of the average pixel intensity $\Delta I$ (Fig.~\ref{GNF-calculation}b). This averaged $\Delta I$ is then multiplied by $l^2$ to give the number density fluctuations $\Delta N$ at the length scale $l$. Note that a second method has also been proposed for calculating number fluctuations, where the standard deviation of bacterial numbers is computed first spatially over different locations in a single time frame and is then averaged over time of different frames \cite{}. Although the two methods lead to the same results when spatial and temporal correlations are small compared with the system size and experiment duration \cite{}, the second method is subject to a systematic error due to potential stable non-uniform light illumination in our experiments. Hence, we use the first method introduced above in our study. Using this method, any non-uniform stable light illumination would result in a zero temporal standard deviation of $I$ and, therefore, would not affect our measurement of true density fluctuations.
\textcolor{red}{I am not sure it is necessary to show Fig.~\ref{GNF-calculation}. Furthermore, it is a little confusing to show two figures with subsystems of different sizes. We can discuss if and how to present the figures better.}



\subsubsection{Normalization of bacterial suspensions of different volume fractions}

Practically, to optimize image qualities, we adjust the exposure time of imaging for suspensions of different $\phi$. Exposure times may affect the proportional constant $b$ in Eq.~\ref{eq:phi-I-relation}, which then introduces a $\phi$-dependent linear constant $b(\phi)$ in Eq.~\ref{eq:n}. Although $b(\phi)$ would not change the scaling exponent of the density fluctuations $\alpha$ at each given $\phi$, it will affect the relative magnitude of $\Delta N$ at different $\phi$. In order to compare the magnitude of density fluctuations at different $\phi$,  we further calibrate and normalize $\Delta I$ for different $\phi$. Specifically, as the calibration, we fix the intensity of illumination and the exposure time and take videos of bacterial suspensions at different $\phi$ under the exact same imaging condition. The calibration results are shown in Fig.~\ref{fig:same-conditions}, where all the curves at different $\phi$ collapse at small length scales. \textcolor{red}{First, why we don't see the diffusive behavior at large $l$ in this plot? Second, I am not sure if I agree with the argument below. Our density-fluctuation measurements show bacterial correlation at small scales even in dilute suspensions. Thus, we cannot detect only single bacterial dynamics at small scales. If that was the case, we should see $\Delta I l^2$ become flat as we discussed before.} The observation is intuitive: small scale density fluctuations is determined primarily by single cell dynamics and is not a strong function on volume fractions. Based on the calibration, we normalize $l^2 \Delta I$ by its value at a fixed small length scale. We choose the small scale at $l = 0.3l_b$ in our study. However, since $l^2 \Delta I$ at different $\phi$ shows the same slope at small $l$, choosing any other small lengths between \textcolor{red}{$XXl_b$} and \textcolor{red}{$XXl_b$} would lead to quantitatively the same results. After the normalization, the density fluctuations measured in our experiments show not only the correct scaling exponent but also the right relative magnitude at different $\phi$.


\subsection{Particle image velocimetry (PIV)}

2D in-plane velocity fields are extracted by Particle Image Velocimetry (PIV) analysis using the openPIV package in Python \cite{Liberzon2020}. %(Fig.~\ref{fig:1}b).
We fix the box size to be 16 $\mu$m, which is larger than the size of a single bacterial body but smaller than the velocity correlation length. A step size of the half of the box size (8 $\mu$m) is used by convention, which sets the spatial resolution of the velocity fields.

\subsection{Energy spectra}
The energy spectra of bacterial suspensions are calculated as follows. First, we apply the built-in Fast Fourier Transform (FFT) function of Python \texttt{numpy.fft} package to convert the discrete velocity field $\bm{v}(\bm{r}) = [v_x(x,y), v_y(x,y)]$ obtained from PIV to the velocity field in the momentum $k$ space $\bm{v_k}(\bm{k}) = [u_k(k_x,k_y),v_k(k_x,k_y)]$. \textcolor{red}{Thanks for clarifying the details on how to do FFT including the step size and $2\pi$ factor. We should certainly document it in your thesis, but I don't think we need to put the details in the paper.} The point-wise kinetic energy density in the $k$-space is then computed as
\begin{multline}
E(k_x, k_y) = \\
\frac{1}{2A}\langle u_k(k_x, k_y)u^*_k(k_x, k_y)+v_k(k_x, k_y)v_k^*(k_x, k_y)\rangle
\end{multline}
where $A$ is the total area of the field of view and $^*$ denotes the complex conjugate. \textcolor{red}{Why is $2A$ not $A$?} The $\langle\cdot\rangle$ indicates an average over multiple images from different times. Finally, the energy spectrum $E(k)$ is obtained by summing up $E(k_x,k_y)$ at a constant $k=(k_x^2+k_y^2)^{1/2}$. \textcolor{red}{I don't think this is exactly what you do. If I am right, you average $E(k_x,k_y)$ in a small ring and then calculate $E(k) = 2\pi k$ times the average $E(k_x,k_y)$. The two approaches are the same mathematically.}

%%The fourier transform of real space velocity $u_k$ and $v_k$ are computed as the following:
%%\begin{itemize}
%%  \item apply the built-in Fast Fourier Transform (FFT) function of Python \texttt{numpy.fft} package to the discrete velocity field $v(x, y)$ (from PIV) to get $V_k(k_x, k_y)$
%%  \item since the FFT is defined as
 %%   $$
 %%   V_k=\sum^{n-1}_{m=0}v_m\exp(-2\pi i \frac{mk}{n})
 %%   $$
 %%   missing the $dx$ counterpart in the continuous Fourier transform, we additionally multiply $d_{step}^2$ to $V_k(k_x, k_y)$ to get the wavenumber domain velocity $v_k(k_x, k_y)$
 %% \item the wavenumber field $k$ corresponding to $v_k(k_x, k_y)$ is obtained by applying the \texttt{numpy.fft.fftfreq} function. Since the definition of FFT introduces a prefactor $2\pi$ to the wavenumber, an additional $2\pi$ is also multiplied to $k$
%%\end{itemize}


% This method can be shown to be equivalent to another commonly used formula, which calculates the Fourier transform of real space velocity spatial correlation functions:
% \begin{equation}
% E^1 = \int \langle \boldsymbol{u}(\boldsymbol{r_0}) \boldsymbol{u}(\boldsymbol{r_0}+\boldsymbol{r}) \rangle
% e^{-i\boldsymbol{k}\cdot\boldsymbol{r}} d\boldsymbol{r}
% \end{equation}
%
% \textcolor{red}{it is good to document the following derivation somewhere (maybe in your thesis), but we don't need to show it in the paper.)}
% We show here how they are equivalent by considering one of the velocity componenet $u$. Starting with the definition of $E$
% \begin{equation}
% \begin{split}
% E(k_x, k_y) &= u_k(k_x, k_y)u_k^*(k_x, k_y)\\
% & = \iint u(x, y)e^{-ik_xx}e^{-k_yy}dxdy\left[\iint u(x', y')e^{-ik_xx'}e^{-ik_yy'}dx'dy'\right]^*\\
% & = \iint u(x, y)e^{-ik_xx}e^{-k_yy}dxdy\iint u^*(x', y')e^{ik_xx'}e^{ik_yy'}dx'dy'\\
% & = \iiiint u(x, y)u(x', y')e^{-ik_x(x-x')}e^{-k_y(y-y')}dxdydx'dy'
% \end{split}
% \end{equation}
% here, we change variable and let $x'' = x - x'$ and $y'' = y - y'$ the original expression can be rearranged into
% \begin{equation}
% \begin{split}
% & \iiiint u(x'+x'', y'+y'')u(x', y')e^{-ik_xx''}e^{-k_yy''}d(x'+x'')d(y'+y'')dx'dy'\\
% & = \iint \left[\iint u(x'+x'', y'+y'')u(x', y')dx'dy'\right] e^{-ik_xx''}e^{-k_yy''} d(x'+x'')d(y'+y'')
% \end{split}
% \end{equation}
% using the definition of velocity correlation function (average all possible pairs over available space):
% \begin{equation}
% \langle u(x, y)u(x+x'', y+y'') \rangle = \frac{\iint u(x'+x'', y'+y'')u(x', y')dx'dy'}{\iint dx'dy'}
% \end{equation}
% we obtain
% \begin{equation}
% \iint dx'dy'\iint \langle u(x, y)u(x+x'', y+y'') \rangle e^{-ik_xx''}e^{-k_yy''} dx''dy''
% \end{equation}
% the first integration is the available space size of velocity field, in this case the system area $A$. Thus,
% \begin{equation}
% E(k_x, k_y) = E^1(k_x, k_y)
% \end{equation}

\begin{figure}[t]
	\begin{center}
		\includegraphics[width=0.46\textwidth]{Figures/local-correlation/v1.pdf}
		\caption[Density autocorrelation]
		{
			Diagram showing the procedures of calculating local coupling between density fluctuations and kinetic energy.
		}
		\label{fig:coupling-calculation}
	\end{center}
\end{figure}

\subsection{Correlation of local density fluctuations and kinetic energies}

To calculate local temporal density fluctuations, we need to approximate instantaneous intensity variations. On the one hand, the time interval for calculating the intensity difference between two frames needs to be smaller than the density correlation time in order to satisfy the instantaneous approximation, which is typically around 2 s (60 frames), as shown in Fig.\ref{fig:spatiotemporal-correlations}. \textcolor{red}{where does the 2 s comes from? The correlation is $4\tau_b$, which is 0.8 s.} On the other hand, the time interval should be sufficient long to suppress the influence of random fluctuations of image intensities in adjacent frames, a factor that always needs to be considered when calculating the derivative of an experimental quantity. In our study, we choose 0.3 s (10 frames) for the local density fluctuation calculation. We do not expect the results to be much different when varying this number from 0.17 to 0.6 s.

\textcolor{red}{I am not clear about the procedure below. What's $F_i(\bm{r})$? Is it the intensity field? Where is the length scale $l = 2.5l_b$ we stated in the main text? Is $2.5l_b = 8$ $\mu$m? The way you calculate kinetic energies is quite different from what I thought (if I understand your procedure right). Why not do PIV on the original images without coarsen-graining and then calculate the kinetic energies with a window by summing up velocity squares in that window. Will that give you the same result? We don't need to show 3f, since it is already in the main text.}
As shown in Fig.~\ref{fig:coupling-calculation}, we first take 10 frames $F_i(\bm{r})$ of consecutive images, where $i=1, 2, ..., 10$. All the 10 frames are coarse-grained by binning $25\times 25$ px$^2$ ($8\times8$ $\mu$m$^2$) windows into a single pixel $f_i(\bm{r})$, as shown in Fig.~\ref{fig:coupling-calculation}b. PIV algorithm is then applied on the first two images to obtain a representative velocity field $v{\bm{r}}$, as in Fig.~\ref{fig:coupling-calculation}c. Since the PIV analysis is done at a step size of 25 pixel, the coarse-graining procedure produces images with dimensions the same as the velocity fields obtained from PIV. We then take the standard deviation of the pixel intensity at each spot over 10 frames to obtain a field of density fluctuations, $\delta N(\bm{r}) = \sqrt{\langle(f_i-\langle f_i\rangle)^2\rangle}$, shown in Fig.~\ref{fig:coupling-calculation}d. The kinetic energy field is obtained by taking square of the magnitudes of PIV result and then divide it by 2, $E_k(\bm{r})= v(\bm{r})^2 / 2 $, as shown in Fig.~\ref{fig:coupling-calculation}e. Finally, the normalized correlaiton between $\Delta N(\bm{r})$ and $E_k(\bm{r})$ is calculated as
%
\begin{equation}
  C = \frac{\langle(\Delta N-\overline{\Delta N})(E_k-\overline{E_k})\rangle}{\sigma_{\Delta N}\sigma_{E_k}}
\end{equation}
%
%$\star$ is the operator standing for cross correlation,
where$\bar\cdot$ means taking the mean, $\sigma$ means the standard deviation, and $\langle\cdot\rangle$ denotes taking the average of all scalars in an array. The cross correlation quantifies the similarity between arrays $N$ and $E_k$. The resulting number takes value from -1 to 1.

\subsection{Density fluctuations in the transient state}

Density fluctuations in the transient state towards active turbulence is calculated using the same method as that in the steady state. Specifically, the procedure described in Sec.~\ref{sec:GNF-calculations} is applied at time $t$ over a time interval of $\Delta t$ during the transition towards active turbulence. We choose $\Delta t = 1.7$ s (50 frames), which is slightly smaller than the density correlation time and thus provides sound statistics. As a comparison, the entire transition takes about 60 s (1800 frames). Thus, $\Delta t$ is also small enough and provides a good temporal resolution to monitor the kinetic process.




% \textcolor{red}{This session is necessary. I have not worked on the revision of the session yet though.}



% We first calculate the autocorrelation functions of density. The basic principle of this calculation is to shift a time series of density at varying steps and calculate the matching between the original series and the shifted series, using the cross correlation scheme as in Eq.~\ref{eq:cross-correlation}. In Fig.~\ref{fig:density-autocorrelation}, we plot the autocorrelation function of density at various volume fractions, covering the range studied in this work. We notice that low volume fraction suspensions display a longer correlation time. The smallest correlation time, according to this measurement, is around 1 second, which corresponds to 30 frames in our videos. Thus, we choose 10 frames as the video length for calculating the local density fluctuations, which is much smaller than the correlation time but still has many enough frames to suppress the random flucutations from imaging.

% \begin{figure}[!]
% \begin{center}
% \includegraphics[width=0.7\textwidth]{figures/density-autocorrelation/v1.pdf}
% \caption[Density autocorrelation]
% {
% \textbf{Density autocorrelation.}
% }
% \label{fig:density-autocorrelation}
% \end{center}
% \end{figure}

% \begin{figure}[t]
% \begin{center}
% \includegraphics[width=0.46\textwidth]{Figures/kinks-in-energy-spectra/v1.pdf}
% \caption[Density autocorrelation]
% {
% \textbf{Energy spectra of the same active turbulence, measured at different PIV step size.}
% }
% \label{fig:kink}
% \end{center}
% \end{figure}



%%\subsection{Kinks in energy spectrum curves}
%%At high $k$ limit, the energy spectra in Fig.~4 show kinks. We show here that such kinks are due to the step size in PIV analysis and in unavoidable when approaching very small length scales. We also note that we choose the smallest step to be the length of single bacterium, so that all lengths above that scale do not show kink.



%%Fig.~\ref{fig:kink} shows the energy spectra of the same sample, but using PIV with different step sizes ranging from 5 to 20 pixels (corresponding to 1.5 to 6 $\mu$m, or 0.5 to 2 $l_b$). Despite the different step sizes used in PIV, the majority of the 4 curves agree with each other well. Kinks happen in the large $k$ limit, and the locations of kinks move towards larger $k$ as step size getting smaller, in consistency with our expectation.

%%\begin{figure}[!]
%%\begin{center}
%%\includegraphics[width=0.7\textwidth]{Figures/GNF-normalization/different-filters.pdf}
%%\caption[Density autocorrelation]
%%{
%%\textbf{GNF curves of the same active turbulence measured with 1 or 2 ND filters below the objective. This is a test experiment showing the effect of imaging condition on the magnitude of GNF curves.}
%%}
%%\label{fig:filter-effect}
%%\end{center}
%%\end{figure}




%%\textcolor{red}{We probably need to document the results somewhere in case referees ask questions, but we don't need to show it in the paper.}

%%\subsection{Does image contrast matter?}

%%Our GNF measurement depends strongly on the image pixel intensities. A natural question is, if we do a simple manipulation of the image, such as autocontrast, would the GNF extracted from the image different? Fig.~\ref{fig:image-contrast} shows the GNF calculation on the same image sequence, but adjusted to different degree of contrast, as shown in Fig.~\ref{fig:image-contrast}a and b. The GNF results are shown in Fig.~\ref{fig:image-contrast}c, the two sequences give rise to exactly the same GNF curve. Note that the normalization described in the previous section is applied here. Otherwise, the magnitude of the curves would be different, while the overall trends are exactly the same.

%%\begin{figure}[!]
%%	\begin{center}
%%		\includegraphics[width=0.7\textwidth]{Figures/image-contrast/v1.pdf}
%%		\caption[Density autocorrelation]
%%		{
%%			\textbf{Test GNF calculation on the same image with different contrast.}
%%			(a) Raw image.
%%			(b) Autocontrasted image.
%%			(c) GNF results.
%%		}
%%		\label{fig:image-contrast}
%%	\end{center}
%%\end{figure}


% \newpage
%
% \section{Supplementary videos}
% Find in ``demo'' folder.
% \subsection{Bacterial turbulence}
% \includegraphics[width=0.5\textwidth]{Figures/video-snapshot/t.jpg}
% \subsection{PIV overlay}
% \includegraphics[width=0.5\textwidth]{Figures/video-snapshot/piv.jpg}

\bibliographystyle{apsrev4-2}
\bibliography{../correlation}




\end{document}


\end{document}
